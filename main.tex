%%% Template originaly created by Karol Kozioł (mail@karol-koziol.net) and modified for ShareLaTeX use

\documentclass[a4paper,11pt]{article}

\usepackage[T1]{fontenc}
\usepackage[utf8]{inputenc}
\usepackage{graphicx}
\usepackage{xcolor}

%\renewcommand\familydefault{\sfdefault}
\usepackage{lmodern}

\usepackage{amsmath,amssymb,amsthm,textcomp}
\usepackage{enumerate}
\usepackage{multicol}
\usepackage{tikz}

\usepackage{geometry}
\geometry{left=25mm,right=25mm,%
bindingoffset=0mm, top=20mm,bottom=20mm}


\linespread{1.3}

\newcommand{\linia}{\rule{\linewidth}{0.5pt}}

% custom theorems if needed
\newtheoremstyle{mytheor}
    {1ex}{1ex}{\normalfont}{0pt}{\scshape}{.}{1ex}
    {{\thmname{#1 }}{\thmnumber{#2}}{\thmnote{ (#3)}}}

\theoremstyle{mytheor}
\newtheorem{defi}{Definition}

% my own titles
\makeatletter
\renewcommand{\maketitle}{
\begin{center}
\vspace{2ex}
{\huge \textsc{\@title}}
\vspace{1ex}
\\
\linia\\
\@author \hfill \@date
\vspace{1ex}
\end{center}
}
\makeatother
%%%

% custom footers and headers
\usepackage{fancyhdr}
\pagestyle{fancy}
\lhead{}
\chead{}
\rhead{}
\lfoot{CID: 01343907}
\cfoot{Time Series Analysis Coursework}
\rfoot{Page \thepage}
\renewcommand{\headrulewidth}{0pt}
\renewcommand{\footrulewidth}{0.5pt}
%

% code listing settings
\usepackage{listings}
\lstset{
    language=python,
    basicstyle=\ttfamily\small,
    aboveskip={1.0\baselineskip},
    belowskip={1.0\baselineskip},
    columns=fixed,
    extendedchars=true,
    breaklines=true,
    tabsize=4,
    prebreak=\raisebox{0ex}[0ex][0ex]{\ensuremath{\hookleftarrow}},
    frame=lines,
    showtabs=false,
    showspaces=false,
    showstringspaces=false,
    keywordstyle=\color[rgb]{0.627,0.126,0.941},
    commentstyle=\color[rgb]{0.133,0.545,0.133},
    stringstyle=\color[rgb]{01,0,0},
    numbers=left,
    numberstyle=\small,
    stepnumber=1,
    numbersep=10pt,
    captionpos=t,
    escapeinside={\%*}{*)}
}

%%%----------%%%----------%%%----------%%%----------%%%

\begin{document}

\title{Time Series Analysis Coursework 2020-2021}

\author{Juliette Limozin, CID: 01343907}

\date{Due 18/12/2020 at 4 pm}

\maketitle

\section*{Question 1}
\subsection*{(a)}

\begin{lstlisting}
def S_AR(f,phis,sigma2):
    """
    INPUTS
    f: a vector of frequencies at which to evaluate the spectral density function.
    phis: the vector [φ1,p, ..., φp,p].
    sigma2: the variance of the white noise.
    OUTPUT:
    S: a vector of values of the spectral density function evaluated at the elements of f.
    """
    if str(type(phis)) != "<class 'numpy.ndarray'>" :
        raise ValueError('Please input a numpy array for phis')
    p = len(phis) #Determine p
    A = np.concatenate((np.array([1]), -phis), axis = 0)
    exps = np.exp(-1j*2*np.pi*np.asarray(range(0,p+1), dtype = 'complex'))
    S = [sigma2/np.dot(A, exps**i) for i in f]
    
    return S
    
    return S
\end{lstlisting}

\subsection*{(b)}

\begin{lstlisting}
def AR2_sim(phis,sigma2,N):
    et = np.random.normal(0, np.sqrt(sigma2), 100+N)
    X = np.zeros(100+N)
    for t in range(2,100+N):
        X[t] = phis[0]*X[t-1]+phis[1]*X[t-2]+et[t]
    
    X = X[100:]
    
    return X
\end{lstlisting}

\newpage

\section*{Question 2}

We observe 500 independent lives over the interval $[x,x+1)$. We assume that individual $i$ comes into view at time $x+a_i$ and is unobserved beyond time $x+b_i$. We want to estimate $\textbf{q} = (_{b_1-a_1}q_{x+a_1}, ... ,_{b_{500}-a_{500}}q_{x+a_{500}})$ by maximising the full likelihood function 
\begin{align*}
    L(\textbf{q}) = \prod_{i = 1}^{500}(_{b_i-a_i}q_{x+a_i})^{d_i}(1-_{b_i-a_i}q_{x+a_i})^{(1-d_i)}
\end{align*}

From lecture notes, we have:
\begin{align*}
   _{b_i-a_i}q_{x+a_i} = \frac{_{b_i}q_x - _{a_i}q_x}{1 - _{a_i}q_x}
\end{align*}
so we can rewrite the likelihood function as
\begin{align*}
    L(\textbf{q}) = \prod_{i = 1}^{500}\left(\frac{_{b_i}q_x - _{a_i}q_x}{1 - _{a_i}q_x}\right)^{d_i}\left(\frac{1 - _{b_i}q_x}{1 - _{a_i}q_x}\right)^{(1-d_i)}
\end{align*}

Estimating the 500 values of $\textbf{q}$ is tedious so the solution is to assume that the probabilities $_tq_x$ vary smoothly as a function of t, $0\leq t \leq 1$. The following three sections provide methods for fractional age adjustment, hence reducing the inference problem to that of estimating the single parameter $q_x$, the estimate being the value $\hat{q_x}$ that maximises $L$.

\subsection*{Uniform distribution of deaths (UDD)}

Assuming UDD, we have 
\begin{align*}
    _tq_x = tq_x, \; 0 \leq t \leq 1
\end{align*}

The likelihood function is then
\begin{align*}
    L(q_x) = \prod_{i = 1}^{500}\left(\frac{b_iq_x - a_iq_x}{1 - a_iq_x}\right)^{d_i}\left(\frac{1 - b_iq_x}{1 - a_iq_x}\right)^{(1-d_i)}
\end{align*}

We estimate $q_x$ by using $optimize$ in R, with the following code:
\begin{lstlisting}[label={list:first},caption=Code to estimate $q_x$ under the UDD assumption]
> data <- Estqx2020
> l_udd <- function(q){
+   return(prod(((data$b*q - data$a*q)/(1-data$a*q))^data$d*((1 - data$b*q)/(1-data$a*q))^{1-data$d}))
+ }
> optimize(l_udd, lower = 0, upper = 1, maximum = TRUE)$maximum
[1] 0.7233351
\end{lstlisting}

Hence $\hat{q_x} = 0.7233351$ using the UDD assumption.

\subsection*{Balducci assumption}

Under the Balducci assumption,
\begin{align*}
   & _tq_x  = \frac{tq_x}{1-(1-t)q_x}, \; 0 \leq t \leq 1 \\
   1 & - _tq_x  = \frac{1-q_x}{1-(1-t)q_x}
\end{align*}

So the rearranged likelihood function is
\begin{align*}
    L(q_x) & = \prod_{i = 1}^{500}\left(\left(\frac{b_iq_x}{1-(1-b_i)q_x} - \frac{a_iq_x}{1-(1-a_i)q_x}\right) \frac{1-(1-a_i)q_x}{1-q_x}\right)^{d_i}\left(\frac{1-q_x}{1-(1-b_i)q_x}\frac{1-(1-a_i)q_x}{1-q_x}\right)^{(1-d_i)} \\
    & = \prod_{i = 1}^{500}\left(\frac{b_iq_x(1-(1-a_i)q_x)}{(1-(1-b_i)q_x)(1-q_x)} - \frac{a_iq_x}{1-q_x}\right)^{d_i}\left(\frac{1-(1-a_i)q_x}{1-(1-b_i)q_x}\right)^{(1-d_i)} \\
    & = \prod_{i = 1}^{500}\left(\frac{b_iq_x(1-(1-a_i)q_x) - a_iq_x(1-(1-b_i)q_x)}{(1-(1-b_i)q_x)(1-q_x)}\right)^{d_i}\left(\frac{1-(1-a_i)q_x}{1-(1-b_i)q_x}\right)^{(1-d_i)}
\end{align*}

We use a similar code as before:
\begin{lstlisting}[label={list:second},caption=Code to estimate $q_x$ under the Balducci assumption]
> l_ba <- function(q){
+   return(prod(((data$b*q*(1 - (1-data$a)*q) - data$a*q*(1 - (1-data$b)*q))/((1 - (1-data$b)*q )*(1-q)))^data$d*((1 - (1-data$a)*q)/(1-(1-data$b)*q))^{1-data$d}))
+ }
> optimize(l_ba, lower = 0, upper = 1, maximum = TRUE)$maximum
[1] 0.778948
\end{lstlisting}

Hence $\hat{q_x} = 0.778948$ using the Balducci assumption.

\subsection*{Constant force of mortality}

Under the constant force of mortality assumption,
\begin{align*}
    _tq_x = 1- (1-q_x)^t, \; 0\leq t \leq 1
\end{align*}

So we get the likelihood function
\begin{align*}
    L(q_x) & = \prod_{i = 1}^{500}\left(\frac{1 - (1-q_x)^{b_i} -1 + (1-q_x)^{a_i}}{1 -1 + (1-q_x)^{a_i}}\right)^{d_i}\left(\frac{1 - 1 + (1-q_x)^{b_i}}{1 - 1 + (1-q_x)^{a_i}}\right)^{(1-d_i)} \\
    & = \prod_{i = 1}^{500}\left(\frac{ - (1-q_x)^{b_i} + (1-q_x)^{a_i}}{ (1-q_x)^{a_i}}\right)^{d_i}\left(\frac{(1-q_x)^{b_i}}{ (1-q_x)^{a_i}}\right)^{(1-d_i)} \\
    & = \prod_{i = 1}^{500}\left(1 - (1-q_x)^{b_i - a_i} \right)^{d_i}\left((1-q_x)^{b_i -a_i}\right)^{(1-d_i)}\\
    & = \prod_{i = 1}^{500}\left( 1- (1-q_x)^{b_i - a_i} \right)^{d_i}(1-q_x)^{(1-d_i)(b_i -a_i)}
\end{align*}

We use a similar code as before:
\begin{lstlisting}[label={list:second},caption=Code to estimate $q_x$ under the constant force of mortality assumption]
> l_cfm <- function(q){
+   return(prod((1-(1-q)^{data$b-data$a})^data$d*(1-q)^{(1-data$d)*(data$b-data$a)}))
+ }
> optimize(l_cfm, lower = 0, upper = 1, maximum = TRUE)$maximum
[1] 0.7552155
\end{lstlisting}

Hence $\hat{q_x} = 0.7552155$ using the constant force of mortality assumption.

\vspace{5ex}

I plotted $_tq_x$ under each assumptions fixing $q_x$ as the three estimates we obtained:

\begin{figure}[h!]
\centering
\begin{minipage}[t]{0.45\linewidth}
    \includegraphics[width = \linewidth]{udd.png}
\end{minipage}
\begin{minipage}[t]{0.45\linewidth}
    \includegraphics[width = \linewidth]{ba.png}
\end{minipage}
\begin{minipage}[t]{0.45\linewidth}
    \includegraphics[width = \linewidth]{cfm.png}
\end{minipage}
\caption{Model fit for Clinicians' model} 
\label{fig5}
\end{figure}

As we can see, because our three estimates are very similar they don't affect the behaviour of $_tq_x$ under each assumptions.

\end{document}
